%-----------------------------------------------
% PACKAGES AND OTHER DOCUMENT CONFIGURATIONS
%------------------------------------------------

\documentclass{resume}

\usepackage{bookmark}
\usepackage{graphicx}
\usepackage{paralist}
\usepackage{multirow}
\usepackage{marvosym}

% Modify document margins
\usepackage[left=0.75in,top=0.6in,right=0.75in,bottom=0.6in]{geometry}

\usepackage{hyperref}
\hypersetup{
  colorlinks=true,
  urlcolor=blue
}

\usepackage[
backend=bibtex,
isbn=false,
url=false,
doi=false,
abbreviate=true,
giveninits=true
]{biblatex}
\addbibresource{bib/resume.bib}
\addbibresource{bib/short-proceedings.bib}


% Logo Commands

\newcommand\githublogo{\raisebox{-1pt}{\includegraphics[height=9pt]{logos/github.pdf}}\ }
\newcommand\linkedinlogo{\raisebox{-1pt}{\includegraphics[height=9pt]{logos/linkedin.pdf}}\ }
\newcommand\emaillogo{\raisebox{-1pt}{\Letter}\ }
\newcommand\homephonelogo{\raisebox{-1pt}{\Telefon}\ }
\newcommand\cellphonelogo{\raisebox{-1pt}{\Mobilefone}\ }
\newcommand*{\github}[1]{\githublogo\href{https://#1}{#1}}
\newcommand*{\linkedin}[1]{\linkedinlogo\href{https://#1}{#1}}
\newcommand*{\email}[1]{\emaillogo\href{mailto:#1}{\nolinkurl{#1}}}
\newcommand*{\homephone}[1]{\homephonelogo{#1}}
\newcommand*{\cellphone}[1]{\cellphonelogo{#1}}


%-----------------------
%   HEADER SECTION
%-----------------------

\name{Georgios ``George'' Balatsouras}

\address{Acheon 23, Agia Paraskevi (Athens 15343) \\ Attica, Greece
  \hfill{\email{gbalats@gmail.com}}
}

\address{\homephone{+30 210 6004683} \hspace{0.2cm}
  \cellphone{+30 6944 205788}
  \hfill{\github{github.com/gbalats}}
}

% \address{
%   \hfill{\linkedin{gr.linkedin.com/pub/george-balatsouras}}
% }


\begin{document}
\newcommand{\mytilde}{\raise.17ex\hbox{$\scriptstyle\mathtt{\sim}$}}


%------------------------------
%     EDUCATION SECTION
%------------------------------

\begin{rSection}{Education}

{\bf Doctor of Philosophy -- PhD,} Static Program Analysis \hfill {2012 -- 2017} \\
University of Athens, Department of Informatics \& Telecommunications \\
Advisor: Prof.~Yannis Smaragdakis
(\href{mailto:smaragd@di.uoa.gr}{\nolinkurl{smaragd@di.uoa.gr}}) \\
Thesis: \emph{Recovering Structural Information for Better Static Analysis}.

{\bf Master of Science,} Computing Systems \hfill {2009 -- 2012} \\
University of Athens, Department of Informatics \& Telecommunications \\
Grade: $9.78 / 10$. Top of graduating class (31 graduates).

{\bf Bachelor of Science,} Computer Science \hfill {2004 -- 2009}  \\
University of Athens, Department of Informatics \& Telecommunications \\
Grade: $8.80 / 10$. Top of graduating class (\mytilde{}50 graduates).

\end{rSection}

%------------------------------
%   WORK EXPERIENCE SECTION
%------------------------------

\begin{rSection}{Experience}

\begin{rSubsection}
  {Hellenic Army Information Technology Support Center (KEPYES)}
  {June 2017 -- present}
  {Software Development Engineer}
  {Athens, Greece}
\item \emph{Description}: Served in the Research and Informatics Corps
  (mandatory military service) with Software Development
  responsibilities. Worked on the development of a Java EE Web
  application with JSF to support existing and add new,
  inter-organizational operations of the Hellenic Army.
\end{rSubsection}

%------------------------------------------------

\begin{rSubsection}
  {University of Athens}
  {July 2011 -- May 2017}
  {Doctoral Researcher, Dept. of Informatics \& Telecommunications}
  {Athens, Greece}
\item \emph{Supervisor}: Prof. Yannis Smaragdakis
  (\href{mailto:smaragd@di.uoa.gr}{\nolinkurl{smaragd@di.uoa.gr}})
\item \emph{FP7 Projects}:
  \begin{inparaenum}[(1)]
  \item
    \href{http://cordis.europa.eu/project/rcn/104361_en.html}{SPADE}
    (``Sophisticated Program Analysis, Declaratively''),
  \item
    \href{http://cordis.europa.eu/project/rcn/95729_en.html}{PADECL}
    (``Advanced Program Analysis Using Declarative Languages'')
  \end{inparaenum}
\item \emph{Description}: Worked on the active development of the
  \textsc{Doop} framework for static analysis of Java
  projects. Developed \texttt{cclyzer}, a new static analysis
  framework for C/C++, using the LogicBlox Datalog engine, the LLVM
  infrastructure, Python, and C/C++.
\end{rSubsection}

%------------------------------------------------

\begin{rSubsection}{LogicBlox, Inc}
  {April 2012 -- August 2012}
  {Software Engineer Intern, Compiler Team}
  {Atlanta, GA, USA}
\item \emph{Supervisor}: Shan Shan Huang
  (\href{mailto:ssh@logicblox.com}{\nolinkurl{ssh@logicblox.com}})
\item \emph{Description}: Contributed to a static analysis tool of
  Datalog programs. Implemented a clone-detection library for Datalog
  code, and a string-concatenation library that reduced space complexity.
% \item Contributed to the active development of \textsc{BloxAnalysis},
%   a tool used to perform static analysis of Datalog programs
%   by representing them in a LogicBlox workspace.
% \item Implemented a clone-detection library using \textsc{BloxAnalysis}
%   that is able to
%   \begin{inparaenum}[(i)]
%   \item trace isomorphic formulas,
%   \item identify redundant rules in Datalog programs, and
%   \item detect alternate indices that can be used in place of original
%     predicates for optimization purposes.
%   \end{inparaenum}
% \item Implemented a string-concatenation library that reduces the
%   space complexity of the intermediate materialized strings from
%   $O(n^2)$ to $O(nlogn)$.
\end{rSubsection}

%------------------------------------------------
\begin{rSubsection}
  {University of Athens}
  {October 2010 -- June 2011}
  {Research Developer, Dept. of Informatics \& Telecommunications}
  {Athens, Greece}
\item \emph{Supervisor}: Prof. Alex Delis
  (\href{mailto:ad@di.uoa.gr}{\nolinkurl{ad@di.uoa.gr}})
\item \emph{Project}: \href{http://pernasvip.di.uoa.gr/index.php}{PERNASVIP}
\item \emph{Description}: Worked on the development of the itinerary
  planner component of a navigation system for visually disabled
  people. Helped achieve great speedups in itinerary planning, by
  optimizing the use of data structures and parallelizing parts of the
  system.
\end{rSubsection}

\end{rSection}

\newpage

\begin{rSection}{Languages and Certificates}
  Greek (native), English (fluent) \\
  Certificate of Proficiency in English, University of Michigan (USA)
  \hfill 2003
\end{rSection}

%----------------------------------
%   TECHNICAL STRENGTHS SECTION
%----------------------------------

\begin{rSection}{Technical Skills}

{\renewcommand{\arraystretch}{1.3}
\begin{tabular}{ @{} >{\bfseries}l @{\hspace{6ex}} l }

Programming Languages
   & C/C++, Java, Python, Prolog, Datalog, Emacs Lisp \\

Build Tools
   & GNU Make, Apache Ant, Apache Maven \\

Version Control
   & Git, Mercurial, Subversion \\

Databases \& ORM
   & SQL, MySQL, LogicBlox, Hibernate \\

Operating Systems
   & Bash Shell Scripting, Linux Command-line, UNIX \\

Editors, IDEs, Debuggers
   & Emacs, IntelliJ IDEA, Eclipse, GCC / G++ / GDB \\

Compiler Engineering
   & LLVM, GNU Bison + Flex, ASM, BCEL \\

% Concurrent Programming
%    & Pthreads, MPI, Java (low-level API, \texttt{java.util.concurrent.*}) \\

Web Development
   & Django, HTML + CSS, JQuery, Bootstrap, JSP / JSF, PrimeFaces \\

Other
   & Docker, \LaTeX{}, LibreOffice, G Suite, AspectJ (AOP), ECLiPSe \\
\end{tabular}}
\end{rSection}

%---------------------------
%   PUBLICATIONS SECTION
%---------------------------

\begin{rSection}{Publications}
  \begin{rSubsection}{}{}{}{}
    \itemsep +0.5pt % increase space between items
  \item \fullcite{mustalias}.
  \item \fullcite{structsens}.
  \item \fullcite{reflection}.
  \item \fullcite{survey}.
  \item \fullcite{introspective}.
  \item \fullcite{jphantom}.
  \item \fullcite{setbased}.
  \end{rSubsection}
\end{rSection}

% \begin{rSection}{Grants and Awards}
%   \begin{rSubsection}{}{}{}{}
%   \item SIGPLAN Professional Activities Committee (PAC) travel grant
%     for ACM OOPSLA 2013.
%   \end{rSubsection}
% \end{rSection}

%-------------------------------------------
%   EXTRA-CURRICULAR ACTIVITIES SECTION
%-------------------------------------------

\begin{rSection}{Activities}
  {\bf Teaching Assistant} \hfill {2009 -- 2014} \\
  University of Athens - Dept. of Informatics \& Telecommunications \\
  Courses: \emph{Advanced OOP, Systems Programming, Operating Systems,
    Artificial Intelligence} \\
\end{rSection}


\begin{rSection}{Hobbies \& Interests}
  reading, music (piano, guitar), role-playing games, Emacs hacking, soccer \\
\end{rSection}


%----------------------------------
%   OPEN-SOURCE PROJECTS SECTION
%----------------------------------

% \begin{rSection}{Projects}
%   \begin{rSubsection}
%     {CClyzer}
%     {Sept 2013 -- April 2017}
%     {Core Developer}
%     {\url{https://github.com/plast-lab/cclyzer}}
%   \item \emph{Description}: \texttt{CCLYZER} is an open-source static
%     analysis tool for C/C++. It accepts whole-program input, compiled
%     to LLVM bitcode, and performs various static analyses, expressed
%     in a fully declarative manner using Datalog. These include a novel
%     \emph{pointer analysis} that may achieve better precision by
%     recovering essential parts of the memory objects' structure.
%     %
%     \vspace{0.3em}
%   \item \emph{Languages / Technologies Used}: C/C++, Python, Datalog,
%     GNU Make, Git, Clang/LLVM.
%   \end{rSubsection}

%   \begin{rSubsection}
%     {JPhantom}
%     {Dec 2012 -- June 2014}
%     {Core Developer}
%     {\url{https://github.com/gbalats/jphantom}}
%   \item \emph{Description}: \texttt{JPhantom} is a static analysis
%     tool that accepts partial Java programs, in the form of Java
%     bytecode (JARs), and transforms them to whole programs with
%     well-formed types, that are more suitable for static analysis.
%     %
%     \vspace{0.3em}
%   \item \emph{Languages / Technologies Used}: Java, Maven, Git, ASM, BCEL.
%   \end{rSubsection}

%   \begin{rSubsection}
%     {Doop}
%     {June 2011 -- Nov 2016}
%     {Developer}
%     {\url{https://bitbucket.org/yanniss/doop}}
%   \item \emph{Description}: \textsc{Doop} is a framework for pointer
%     analysis of Java programs. It implements a range of algorithms,
%     including context insensitive, call-site sensitive, and
%     object-sensitive analyses.
%     %
%     \vspace{0.3em}
%   \item \emph{Languages / Technologies Used}: Java, Datalog, Ant,
%     Mercurial, Bash.
%   \end{rSubsection}
% \end{rSection}


\end{document}
